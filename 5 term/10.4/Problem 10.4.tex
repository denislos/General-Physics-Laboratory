\documentclass[12pt]{article}

\usepackage{geometry}
\usepackage[utf8]{inputenc}
\usepackage[T2A]{fontenc}
\usepackage[russian]{babel}
\usepackage{graphicx}
\usepackage{caption}
\usepackage{amssymb, gensymb, amsmath}
\usepackage{mathrsfs}
\usepackage{array, colortbl}
\usepackage{multicol}

\title{{\bf Лабораторная работа 10.\, 4. \\ Магнитные моменты лёгких ядер}}
\author{Лось Денис (группа 618)}
\date{7 декабря 2018}

\begin{document}

\maketitle

\paragraph*{Цель работы: } вычислить магнитные моменты протона, дейтрона и ядра фтора на основе измерения их $g$-факторов методом ядерного магнитного резонанса (ЯМР). Сравнить полученные данные с вычесленными магнитными моментами на основе кварковой модели адронов и одночастичной оболочечной модели ядер.

\section*{Теоритическая часть}
\par
	Полный момент ядра:
\[
	I = L + S
\]
где $L$ --- полный орбитальный момент нуклонов, $S$ --- собственная часть количества движения, спин.
\par
	Полный момент количества движения изолированной системы (ядра) принимает целые или полуцелые значения в единицах $\hat{h}$. Для чётного числа нуклонов $I$ --- целое, а для нечётного --- полуцелое.
\par
	Отношение дипольного момента $\mu$ ядра к механическому моменту называется гиромагнитным соотношением:
\[
	\gamma = g \gamma_0
\]
, где $g$ --- фактор Ланде, а за единицу $\gamma_0$ принимается гиромагнитное отношение для орбитального движения электрона в атоме:
\[
	\gamma_0 = - \frac{e}{2 m_e c}
\]
\par
	Аналогично, в ядерной физике:
\[
	\gamma_n = \frac{e}{2 M c}
\]
\par
	Магнитный момент ядра:
\[
	\mu = \gamma_n \hat{h} I = \gamma_n \mu_n I
\]
\par
	Способы определения углового момента ядра:
\begin{enumerate}
	\item 
		Cверхтонкая структура оптических спектров
	\item
		Чередование интенсивностей в полосатых спеткрах двух-атомных молекул с тождественными ядрами.
	\item
		Ядерные реакции, $\beta$ и $\alpha$ распады.
	\item
		Ядерный магнитный резонанс. ЯМР --- это резонансное поглощение электромагнитной энергии в веществах, обусловленное ядерным перемагничиванием. ЯМР наблюдается в постоянном магнитном поле $H_0$ при одновременном воздействии на образец радиочастотного магнитного поля, перпендикулярного $H$, и обнаруживается по поглощению излучения.
	\par
		В магнитном поле уровни ядра расщепляются и под действием внешнего высокочастотного поля могут происходить электромагнитные переходы между компонентами расщепившегося уровня, это явление носит резонансный характер. Различие по энергии между двумя соседними компонентами:
\[
	\Delta E = \text{г}_\text{яБ} \mu_\text{я} B_0
\]
\par
	Частота квантов:
\[
	f_0 = \frac{\Delta E}{h} = \frac{\gamma_\text{я} \mu_\text{я} B_0}{h}
\]

\section*{Ход работы и результаты исследования}

\subsection*{Образец 3: вода}
\begin{align*}
	f_0 &= \left(10.234 \pm 0.106\right) \, \text{МГц} \\
	B &= 238 \, \text{мТ} \\
	g &= \left(5.64 \pm 0.06\right) \\
	\mu &= \left(2.82 \pm  0.03\right) \cdot \mu_\text{я}
\end{align*}

\subsection*{Образец 1: резина}
\begin{align*}
	f_0 &= \left(9.967 \pm 0.107\right) \, \text{МГц} \\
	B &= 233 \, \text{мТ} \\
	g &= \left(5.61 \pm 0.06\right) \\
	\mu &= \left(2.81 \pm  0.03\right) \cdot \mu_\text{я}
\end{align*}

\subsection*{Образец 2: тефлон (ядра фтора)}
\begin{align*}
	f_0 &= \left(9.214 \pm 0.109\right) \, \text{МГц} \\
	B &= 228 \, \text{мТ} \\
	g &= \left(5.30 \pm 0.06\right) \\
	\mu &= \left(2.65 \pm  0.03\right) \cdot \mu_\text{я}
\end{align*}


\end{enumerate}

\end{document}